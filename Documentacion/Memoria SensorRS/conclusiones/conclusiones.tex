% Este archivo es parte de la memoria del proyecto fin de carrera
% de Manuel López Urbina. Protegida bajo la licencia GFDL.
% Para más información, la licencia completa viene incluida en el
% fichero fdl-1.3.tex

% Copyright (C) 2018 Manuel López Urbina

\newpage

\chapter{Conclusiones}
\label{chap:conclusiones}

La elaboración de este trabajo ha resultado muy gratificante a nivel personal. Uno de los motivos principales ha sido la necesidad de trabajar en numerosas áreas de 
conocimiento entre las que encontramos, por un lado la programación, ya que ha sido necesaria la programación del microcontrolador que incorpora la placa Arduino, 
concretamente el ATmega2560. Por otra parte se ha realizado la programación de la placa Raspberry Pi, módulo encargado de la comunicación por puerto serie con la placa Arduino y la 
transmisión de los datos al servidor.\\

Por otro lado, la elaboración del vehículo robótico me ha permitido ampliar conocimientos en áreas más relacionadas con la electrónica, un área bastante desconocida para mí. 
Donde se ha trabajado, entre otras áreas, en lo referente a la transmisión de señales por puerto serie, la emisión de pulsos PWM junto con la utilización de multitud de sensores.\\

Entre los elementos desarrollados podemos destacar:\\

\begin{itemize}
 \item La elaboración de un vehículo robótico haciendo uso de una Raspberry Pi 3 Model B.
 \item Programación del microcontrolador Atmega2560 alojado en una placa Arduino.
 \item Comunicación por puerto serie entre la placa Arduino y Raspberry Pi
 \item Realización de conexiones entre elementos y montaje del vehículo.
 \item Transmisión de gran cantidad de datos entre cliente servidor y servidor cliente. Streaming de vídeo y audio, emisión de comandos entre otros datos.\\
 \item Incorporación de algún sistema para el geoposicionamiento y escaneado de habitaciones, obstáculos, etc.
\end{itemize}

Pienso que el presente trabajo puede ser de gran utilidad permitiendo liberar a los trabajadores de aquellas tareas especialmente peligrosas para la salud y la integridad física en diversas situaciones 
como exploración de zonas peligrosas o de difícil acceso, búsqueda de personas, etc ya que el vehículo desarrollado permite introducirlo de manera remota en zonas inaccesibles 
o peligrosas para las personas.\\

Una vez presentado podré continuar añadiendo mejoras y muchas otras cosas que tengo pensadas y que, posiblemente, se realicen a modo proyecto personal y ocio.\\

\section{Mejoras futuras}
\label{sec:mejoras_futuras}

La aplicación puede mejorarse en diversos aspectos. A continuación, se citan algunas de las mejoras que pueden llevarse a cabo:

\begin{itemize}
  \item Incrementar el número de sensores en el sistema que permitan captar nuevos parámetros o situaciones.
  \item Dotar del sistema de la posibilidad de operar con más canales de conectividad distintas al WiFi como 4G, buetooth, etc.
  \item Incorporar una cámara con posibilidad de visión 360 grados con un sistema de servomotores operable por el usuario.
  \item Añadir autenticación por un sistema de certificados por clave pública y privada que garantice que la conectividad entre dispositivo robótico y usuario es la correcta.
\end{itemize}

