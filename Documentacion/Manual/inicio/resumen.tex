% Este archivo es parte de la memoria del proyecto fin de carrera
% de Manuel López Urbina. Protegida bajo la licencia GFDL.
% Para más información, la licencia completa viene incluida en el
% fichero fdl-1.3.tex

% Copyright (C) 2017 Manuel López Urbina

\section*{Resumen}
\label{resumen}

RobotUI es un proyecto web elaborado con el framework Sails.js cuya intención principal es la de permitir que usuarios compartan sus dispositivos robóticos a través de internet.\\

Para ello incorpora un asistente mediante el cual, cualquier usuario sin conocimientos previos de programación, pueda elaborar una interfaz para el control de su dispositivo robótico, personalizada y
adaptada a sus necesidades de control y a las características del dispositivo en cuestión.\\

Dicha interfaz permitirá realizar el control de los mencionados dispositivos además de permitir que otros usuarios entren en las salas o canales donde podrán visualizar el manejo 
que realiza un usuario de su dispositivo robótico en tiempo real. Esto es posible gracias a una especie de retransmisión o comunicación (Empleo de Websockets) donde, si el robot dispone de cámara, obtendrán en todo momento las imágenes captadas 
junto con los diferentes comandos accionados por el usuario que está realizando el control.\\

Por otra parte, el usuario propietario del robot puede permitir el control de su robot a otros usuarios. todo ello con una correcta gestión de permisos en las que se deniegan o permiten 
las diferentes modalidades de acceso a cada usuario de la aplicación.\\

Con ello, lo que se busca es abrir el acceso de aquellos proyectos robóticos a la red de redes permitiendo el \emph{RobotSharing} entre usuarios.\\


\textbf{Palabras clave:} Internet, aplicación web, robótica, robots, interfaz de usuario, streaming de vídeo, control remoto, robot sharing, Sails.js, Node.js, Socket.io, Websockets, compartir, 
asistente, tiempo real, Raspberry Pi.
