% Este archivo es parte de la memoria del proyecto fin de carrera
% de Manuel López Urbina. Protegida bajo la licencia GFDL.
% Para más información, la licencia completa viene incluida en el
% fichero fdl-1.3.tex

% Copyright (C) 2017 Manuel López Urbina

\chapter{Comentarios finales}
\label{chap:conclusiones}

\section{Presupuesto}


\begin{table}[H]
  \begin{center}
    \begin{tabular}{|p{8cm}|p{2cm}|p{2cm}|p{2cm}|}
      \hline
      \vspace{+0.2in}{\textbf{Descripción}} & {\textbf{Unidades}} & {\textbf{Precio \EUR{}}} & {\textbf{Total \EUR{}}}\\
      \hline
      \vspace{+0.2in}{Raspberry Pi 3 Modelo B} &  \vspace{+0.2in}{1} &  \vspace{+0.2in}{38,7} &  \vspace{+0.2in}{38,7}\\
      \hline
      \vspace{+0.2in}{Lipo batería (3.7v, 600mAh Lipo) } & \vspace{+0.2in}{1} & \vspace{+0.2in}{13,99} & \vspace{+0.2in}{13,99}\\
      \hline
      \vspace{+0.2in}{Indicador Tester de baterías Lipo} & \vspace{+0.2in}{1} & \vspace{+0.2in}{8,90} & \vspace{+0.2in}{8,90}\\
      \hline      
      \vspace{+0.2in}{Cargador baterías Lipo} & \vspace{+0.2in}{1} & \vspace{+0.2in}{21,90} & \vspace{+0.2in}{21,90}\\
      \hline
      \vspace{+0.2in}{Vehículo Radiocontrol} & \vspace{+0.2in}{1} & \vspace{+0.2in}{54} & \vspace{+0.2in}{95}\\
      \hline
      \vspace{+0.2in}{Droplet Digital Ocean} & \vspace{+0.2in}{2} & \vspace{+0.2in}{5/mes} & \vspace{+0.2in}{10}\\
      \hline        
      \vspace{+0.2in}{Horas de programación} & \vspace{+0.2in}{350} & \vspace{+0.2in}{50/hora} & \vspace{+0.2in}{17500}\\
      \hline
    \end{tabular}
  \end{center}
\end{table}

\begin{table}[H]
  \begin{flushright}
    \begin{tabular}{p{8cm}p{2cm}}
      \vspace{+0.1in}\textbf{Total bruto:} &\vspace{+0.1in}{\EUR{17688,49}}\\
      \vspace{+0.1in}\textbf{I.V.A. \%: } & \vspace{+0.1in}{21\%}\\
      \vspace{+0.2in}\textbf{Total presupuesto:} & \vspace{+0.2in}{\EUR{21403,07}}\\
    \end{tabular}
  \end{flushright}
\end{table}


\section{Conclusiones}

La elaboración de este proyecto ha resultado muy gratificante a nivel personal. Uno de los motivos principales ha sido la necesidad de trabajar en numerosas áreas de conocimiento entre las que encontramos, por un lado la programación web, haciendo uso del framework Sails js, junto con el empleo de una base de datos no relacional. 
Todo ello combinado con la robótica. Algunas de las plataformas mencionadas eran desconocidas al inicio del desarrollo de proyecto y han sido adquiridas tras una amplia labor de investigación.\\

Entre los elementos desarrollados se destaca:

\begin{itemize}
 \item La elaboración de un vehículo de pruebas haciendo uso de una Raspberry Pi+
 \item Aprendizaje a la utilización del framework Sails.js
 \item Aprendizaje al trabajo con eventos en tiempo real mediante el empleo de WebSockets, tecnología nunca utilizada por mí hasta la fecha
 \item Empleo de una base de datos no relacional como Mongo DB.
 \item Transmisión de gran cantidad de datos entre cliente servidor y servidor cliente. Streaming de vídeo y emisión de comandos entre otros datos.
\end{itemize}


Pienso que el resultado final del proyecto es ideal para aquellas personas aficionadas a la robótica y programación proporcionando una herramienta sea utilizable por la gran comunidad poseedora del de cualquier proyecto robótico y que puedan compartirlo con el resto del mundo.\\

Una vez presentado podré continuar añadiendo mejoras y muchas cosas que tengo pensadas y que, posiblemente, se realicen para el proyecto del máster de Ingeniería de Sistemas y Computación que me encuentro realizando en la actualidad.

\section{Mejoras futuras}

La aplicación puede mejorarse en diversos aspectos. A continuación, se citan algunas de las mejoras que pueden llevarse a cabo:

\begin{itemize}

\item Incorporación de advertencias acústicas tras la detección de una señal de tráfico.

\item Mejora del diseño de la interfaz gráfica.

\item Permitir la definición de las teclas de control del teclado e incorporación de dispositivos tales como gamepads o joysticks.

\item Elaborar un generador de código para la exportación en los dispositivos robóticos, reduciendo por tanto las labores de programación.

\end{itemize}


