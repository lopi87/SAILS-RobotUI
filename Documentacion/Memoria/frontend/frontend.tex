% Este archivo es parte de la memoria del proyecto fin de carrera
% de Manuel López Urbina. Protegida bajo la licencia GFDL.
% Para más información, la licencia completa viene incluida en el
% fichero fdl-1.3.tex

% Copyright (C) 2017 Manuel López Urbina

\chapter{ Desarrollo de la interfaz }
\chaptermark{Interfaz}

\label{chap:desarrollo-frontend}

\section {Elementos de la interfaz}

Para la elaboración de parte referente la capa de presentación\footnote{ Terminología también conocida como Front-End. Son todas aquellas tecnologías que corren del lado del cliente, es decir, todas aquellas tecnologías que corren del lado del navegador web, generalizándose más que nada en tres lenguajes, Html , CSS Y JavaScript, normalmente el FrontEnd se encarga de estilizar la página de tal manera que la página pueda quedar cómoda para la persona que la utiliza transmitiendo una experiencia de usuario cómoda. En contraposición con el back-end que engloba el motor de la aplicación.} de la aplicación se ha empleado Bootstrap, un potente framework CSS que fue creado por Twitter para simplificar el proceso de maquetación. Este framework CSS presenta multitud de ventajas:

\begin{itemize}
  \item Permite crear interfaces adaptables a los diferentes navegadores, tanto de escritorio como tablets o móviles con distintas escalas y resoluciones.

  \item Se integra perfectamente con las principales librerías Javascript, entre ellas JQuery.

  \item Ofrece un diseño sólido usando LESS y estándares como CSS3/HTML5.

  \item Es un framework ligero y fácilmente integrable en la mayoría de proyectos.

  \item Funciona con los navegadores más populares.

  \item Dispone de distintos layouts predefinidos con estructuras fijas a 940 píxeles de distintas columnas o diseños fluidos.
\end{itemize}

Por otra parte, dada la cantidad de efectos dinámicos necesarios en la aplicación junto con múltiples llamadas Ajax, se ha utilizando jQuery que es un framework Javascript que facilita toda esta
labor haciéndola compatible con los navegadores más comúnmente utilizados.\\

Se consideró antes de comenzar a definir los diferentes estilos y características desarrollo de la interfaz gráfica, una serie de elementos mínimos imprescindibles que dicha interfaz debería incorporar.\\

Dada las características del proyecto, era necesario proporcionar a la aplicación de un entorno gráfico sencillo, intuitivo pero a la vez funcional teniendo como principal objetivo ofrecer al usuario, tras 
una visión rápida, la localización de los diferentes elementos para control del robot y deducir el funcionamiento general de la aplicación.\\

Las vistas principales desarrolladas en la aplicación son los siguientes:

\begin{itemize}
 \item Vista principal de la aplicación.
 \item Panel de administración para el visionado de los usuarios conectados en tiempo real.
 \item Panel de administración para el visionado de los robots conectados en tiempo real.
 \item Formularios de registro de usuarios y robots.
 \item Panel de creación y edición de interfaces de control.
 \item Panel de control de robots.
 \item Panel de visualización de robots.
 \item Índice de robots disponibles.
\end{itemize}


El desarrollo de todos los componentes anteriormente mencionados se han ido realizando paralelamente al desarrollo de la parte Backend\footnote{El BackEnd, en contraposición al FrontEnd
es el área dedicada a la parte lógica de un sitio web, el BackEnd engloba toda la parte no visible para el usuario, excluyendo toda la parte de diseño o de elementos gráficos.} correspondiente.\\
