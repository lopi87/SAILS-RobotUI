% Este archivo es parte de la memoria del proyecto fin de carrera
% de Manuel López Urbina. Protegida bajo la licencia GFDL.
% Para más información, la licencia completa viene incluida en el
% fichero fdl-1.3.tex

% Copyright (C) 2017 Manuel López Urbina

\chapter[Requisitos]{Especificación y análisis de requisitos}
\label{chap:requisitos}

Para el caso que nos concierne en el proyecto, dentro del marco de investigación que define la totalidad de la infraestructura, la funcionalidad principal del mismo será:

\begin{itemize}
\item Definir los pasos para dar de alta un dispositivo robótico en el sistema.
\item Una vez configurado el dispositivo, configurar la interfaz de control con las acciones de control específicas.
\item Realizar un sistema de monitorización y visualización para los usuarios espectadores en tiempo real.
\item Sistema de gestión de base de datos en donde se encuentren los datos de la aplicación recogidos.
\item Panel de administración donde visualizar la información de los usuarios conectados y dispositivos en uso en tiempo real.
\end{itemize}

\section[Especificación]{Especificación de los requisitos}

En esta etapa del modelado de requisitos se captura el propósito general del sistema:

\begin{itemize}
\item Se analiza qué debe hacer el sistema.
\item Se obtiene una versión contextualizada del sistema.
\item Identifica y delimita el sistema.
\item Se determinan las características, cualidades y restricciones que debe satisfacer el sistema.
\end{itemize}

\subsection{Requisitos funcionales}

Los requisitos funcionales que se han obtenido en el sistema son los siguientes:

\begin{itemize}
\item Ser una herramienta multiplataforma y que permita a cualquier usuario definir sus propias interfaces para el control de robots.
\item Dotar de funcionalidad gráfica que permita en tiempo real con mecanismos visuales (en web) visualizar el control de dispositivos robóticos por parte de otros usuarios, modo espectador de la aplicación.
\item Proporcionar un sistema de streaming de vídeo para la difusión de imágenes a los usuarios espectadores procedentes de los robots dispongan de cámara.
\item Implementar un panel de administración para la visualización de usuarios y dispositivos conectados en tiempo real.
\end{itemize}

\subsection{Requisitos no funcionales}

Los requisitos no funcionales son aquellos que describen cualidades o restricciones del sistema que no se relacionan de forma directa con el comportamiento funcional del mismo. A continuación se especifican los más importantes del sistema:
\begin{itemize}
\item No requiere un conocimiento específico del sistema una vez puesto en funcionamiento.
\item La aplicación tendrá manual de uso.
\item La base de datos estará implementada en un lenguaje objeto no relacional como MongoDB.
\item La aplicación estará realizada en el lenguaje de programación Python.
\item La interfaz debe reflejar claramente la distinción entre las distintas partes del sistema.
\item El sistema se desplegará sobre una versión GNU Linux Debian 8 Jessie.
\item El código fuente de la aplicación seguirá un estilo uniforme y normalizado para todos los módulos del mismo.
\end{itemize}
