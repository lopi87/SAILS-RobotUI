% Este archivo es parte de la memoria del proyecto fin de carrera
% de Manuel López Urbina. Protegida bajo la licencia GFDL.
% Para más información, la licencia completa viene incluida en el
% fichero fdl-1.3.tex

% Copyright (C) 2017 Manuel López Urbina

\newpage


\chapter{ Pruebas }
\label{chap:pruebas}

La realización de pruebas de software, es una de las fases del ciclo del software y
permiten la identificación de posibles fallos en la implementación, calidad o usabilidad
de la aplicación.\\

\section{Plan de pruebas}

Debido a la arquitectura de RobotUI y a la metodología basada en desarrollo incremental que
se ha seguido en el desarrollo, las diferentes partes han ido siendo probadas de forma
independiente. Aun así, la aplicación ha sido sometida a las siguientes pruebas:\\


\begin{description} 

\item [Pruebas de configuración:]\\

\begin{itemize}
    \item[]
 \item Se controla que el sistema de gestión de permisos funciona correctamente.
\end{itemize}


\item [Pruebas de navegación:]\\

\begin{itemize}
    \item[]
 \item Se comprueba que se realiza una navegación correcta por toda la aplicación.
\end{itemize}


\item [Pruebas de interfaz:] \\

\begin{itemize}
    \item[]
 \item Se comprueba que los estilos de la aplicación son uniformes y correctos.
 \item Se comprueba que se actualiza de forma satisfactoria la imagen de estado de un usuario, \emph{online - offline}.
 \item Se comprueba que se actualiza de forma satisfactoria la imagen de estado de un robot, \emph{ libre - ocupado} y \emph{online - offline}.
 \item Se comprueba que se actualiza de forma satisfactoria el panel de usuarios realizando el seguimiento de un robot, (entrada y salidas de usuarios de la sesión).
\end{itemize}


\item [Pruebas de funcionamiento:]\\

\begin{itemize}
    \item[]
\item Se controla el estado \emph{online} u \emph{offline} de un robot.
\item Se controla el estado \emph{libre} u \emph{ocupado} de un robot.
\item Se controla el estado \emph{online} u \emph{offline} de un usuario.
\item Se realiza el control de un robot correctamente.
\item Se realiza el seguimiento de un robot correctamente.
\item Se realiza la configuración de una interfaz para un robot determinado.
\item Se realiza el registro de un robot correctamente.
\end{itemize}



\end{description}