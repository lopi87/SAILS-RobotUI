% Este archivo es parte de la memoria del proyecto fin de carrera
% de Manuel López Urbina. Protegida bajo la licencia GFDL.
% Para más información, la licencia completa viene incluida en el
% fichero fdl-1.3.tex

% Copyright (C) 2017 Manuel López Urbina

\chapter{Conceptos básicos}
\label{chap:conceptos-básicos}



\section{WebSocket}
\label{sec:def-websocket}

WebSocket es una tecnología que proporciona un canal de comunicación bidireccional y full-duplex sobre un único socket TCP.
Está diseñada para ser implementada en navegadores y servidores web, pero puede ser utilizada por cualquier aplicación cliente/servidor.\\

La API de WebSocket está siendo normalizada por el W3C, mientras que el protocolo WebSocket ya fue normalizado por la IETF\footnote{Internet Engineering Task Force (IETF) (en español, Grupo de Trabajo de Ingeniería de Internet)
es una organización internacional abierta de normalización, que tiene como objetivos el contribuir a la ingeniería de Internet, actuando en diversas áreas, como transporte, encaminamiento, seguridad.
Se creó en los Estados Unidos, en 1986. Es mundialmente conocido porque se trata de la entidad que regula las propuestas y los estándares de Internet, conocidos como RFC.} como el RFC 6455.\\

Debido a que las conexiones TCP comunes sobre puertos diferentes al 80 son habitualmente bloqueadas por los administradores de redes, el uso de esta tecnología proporcionaría una solución
a este tipo de limitaciones proveyendo una funcionalidad similar a la apertura de varias conexiones en distintos puertos, pero multiplexando diferentes servicios WebSocket sobre un único
puerto TCP a costa de una pequeña sobrecarga del protocolo.


\section{Streaming}
\label{sec:def-streaming}


La retransmisión (en inglés streaming, también denominado transmisión) es la distribución digital de contenido multimedia a través de una red de computadoras, 
de manera que el usuario utiliza el producto a la vez que se es descargado. La palabra retransmisión se refiere a una corriente continua que fluye sin interrupción, habitualmente audio o vídeo, aplicándose la difusión 
de vídeo en el presente proyecto. \\

Este tipo de tecnología funciona mediante un búfer de datos que va almacenando el flujo de descarga en la estación del usuario para inmediatamente mostrarle el material descargado. Esto se contrapone al mecanismo de
descarga de archivos, que requiere que el usuario descargue los archivos por completo para poder acceder al contenido.\\

La retransmisión requiere de una conexión por lo menos de igual ancho de banda que la tasa de transmisión del servicio. La retransmisión de vídeo por Internet se popularizó a fines de la década de 2000, 
cuando la contratación del suficiente ancho de banda para utilizar estos servicios en el hogar se hizo lo suficientemente barato.