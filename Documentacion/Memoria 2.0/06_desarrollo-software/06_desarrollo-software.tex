\newpage

\chapter{Desarrollo software }
\chaptermark{desarrollo}
\label{chap:desarrollo-software}

\section{Metodología de desarrollo}

Este proyecto ha sido elaborado empleando una metodología de desarrollo basada en el modelo de desarrollo incremental para la parte software referente a todos los subsistemas web y una metodología de 
desarrollo en cascada para el desarrollo de la parte software referente al robot de pruebas.\\

El modelo de desarrollo incremental proporciona una serie de características que lo hacen idóneo para este proyecto. Dicho modelo se basa en la filosofía de construir 
e ir incrementando las funcionalidades del sistema mediante el desarrollo de los diferentes módulos. Esto permite ir aumentando gradualmente las capacidades del software. \\

Dicha metodología de desarrollo resulta especialmente útil en las siguientes situaciones:\\

\begin{itemize}
 \item Facilita el desarrollo permitiendo a cada miembro del equipo desarrollar un módulo particular. En el caso del presente proyecto me ha permitido desarrollar un módulo tras otro de una manera secuencial.
 \item Es similar al ciclo de vida en cascada aplicándose un ciclo en cada nueva funcionalidad del programa.
 \item A final de cada ciclo se entrega el software al cliente. En el caso que compete a este proyecto se mantenía una reunión con el director del proyecto para su aprobación.
\end{itemize}

Centrándonos nuevamente en el desarrollo del proyecto, los motivos que llevaron a cabo la elección de un modelo de desarrollo incremental viene dada por la necesidad de simplificar e ir
desarrollando de una forma gradual y modularizada debido a la extensión del proyecto. Más si cabe que el equipo de desarrollo solo consta de una persona.\\

Por otro lado, para el desarrollo del vehículo de pruebas y por su simplicidad, se ha optado por un desarrollo en cascada. El modelo de desarrollo en cascada resulta adecuado en situaciones
en las que:\\

\begin{itemize}
 \item Se dispone de unos requisitos claros y precisos.
 \item El sistema a desarrollar es de pequeña envergadura.
 \item Las tecnologías utilizadas son conocidas por los desarrolladores.
\end{itemize}

Siendo precisamente éstas las características del proyecto del vehículo a desarrollar puesto que se trata de un desarrollo de pequeño tamaño y las herramientas empleadas ya me resultaban conocidas
tras la realización de otros trabajos previos.\\

Por tanto el proyecto queda distribuido en los siguientes subsistemas:\\

\begin{figure}[H]
  \begin{center}
    \includegraphics[scale=.6]{diagramas/subsistemas.png}
  \end{center}
  \caption{Subsistemas existentes en el proyecto junto con el modelo de ciclo de vida
utilizado para su desarrollo.}
  \label{website:pagina-principal}
\end{figure}

\subsubsection{Diagrama de casos de uso}

Una vez analizados los componentes hardware principales a utilizar, pasamos al análisis de los diferentes requerimientos funcionales para el vehículo a desarrollar.

\begin{figure}[H]
  \begin{center}
    \includegraphics[scale=.5]{diagramas/casos-uso-robot.png}
  \end{center}
  \caption{Diagrama de casos de uso para la interacción con el robot.}
  \label{diagram:caso-uso}
\end{figure}

\subsubsection{Estados del robot}

Otro de los requerimientos es que el robot deberá de responder a una serie de estados de tal modo que en la aplicación se conozca las diferentes situaciones en la que se pueda encontrar un robot.
En el autómata diseñado encontramos un modo desactivado, modo de espera y modo de conexión. El autómata representativo es el siguiente:\\

\begin{figure}[H]
  \begin{center}
    \includegraphics[scale=0.6]{imagenes/robot/automata-estados.png}
  \end{center}
  \caption{Autómata representativo de los diferentes estados del robot.}
  \label{figura:automata-estados}
\end{figure}

Donde:\\

\begin{itemize}
  \item Estado q0 inicial y final se corresponde con el estado apagado.
 \item Estado q1 se corresponde con el estado en escucha.
 \item Estado q2 se corresponde con el estado en funcionamiento.
\end{itemize}


\section{Software de control}
 
Para la programación del robot se ha empleado el lenguaje de programación JavaScript en un entorno de ejecución Node.js. A continuación se describirá aquellos aspectos más importantes
referentes al código desarrollado para el control del robot.\\

Primeramente se ha realizado la carga de librerías necesarias, entre ellas encontramos:

\begin{itemize}
 \item \emph{pigpio}: Módulo para la comunicación y control de los pines GPIO.
 \item \emph{child process}: El módulo child\_process proporciona la capacidad de generar procesos secundarios. Se ha empleado para la captura de vídeo mediante el lanzado de comandos ffmpeg.
 \item \emph{socket.io}: Biblioteca que establece enlaces bidireccionales en tiempo real y en comunicación basada por eventos.
 \item \emph{Arduino programming language}  está basado en C++ y aunque la referencia para el lenguaje de programación de Arduino está en \href{http://arduino.cc/en/Reference/HomePage}, también es posible usar comandos estandar de C++ en la programación de Arduino.
\end{itemize}


\subsection{Entrada/Salida}

En segundo lugar se han definido los diferentes pines GPIO a utilizar y si serán empleados como pines de entrada o de salida:

\begin{table}[H]
  \begin{center}
    \begin{tabular}{|p{2.5cm}|p{2.5cm}|p{4.5cm}|}
      \hline
      {\textbf{GPIO}} & \textbf{ Modo } & \textbf{ Control }\\
      \hline
      {\textbf{ 2 }} & { OUTPUT } & { Motores lado izquierdo }  \\
     \hline
      {\textbf{ 3 }} & { OUTPUT } & { Motores lado izquierdo } \\
      \hline
      {\textbf{ 17 }} & { OUTPUT } & {  Motores lado derecho } \\
      \hline
      {\textbf{ 27 }} & { OUTPUT } & { Motores lado derecho } \\
     \hline   
    \end{tabular}
  \end{center}
\caption{ Configuración establecida para los puertos GPIO. }
\end{table}


Si para el vehículo que deseemos programar resultan necesarios más pines para la utilización de servos, sensores o cualquier otro elemento, tan solo debemos inicializarlos e indicar si van a ser pines de entrada
o de salida. Si existen dudas al respecto se puede acceder a la documentación de la biblioteca \emph{pigpio} en el siguiente enlace: \url{https://www.npmjs.com/package/pigpio}.
Para el caso de este proyecto, la inicialización de los pines se ha realizado mediante las siguientes instrucciones:

\begin{lstlisting}[language=JavaScript]
  // Carga del módulo.
  var Gpio = require('pigpio').Gpio;

  // Pines utilizados. Motores izquierdos: 2 y 3, motores derechos: 17 y 27
  var gpio2 = new Gpio(2, {mode: Gpio.OUTPUT}),
    gpio3 = new Gpio(3, {mode: Gpio.OUTPUT}),
    gpio17 = new Gpio(17, {mode: Gpio.OUTPUT}),
    gpio27 = new Gpio(27, {mode: Gpio.OUTPUT});
\end{lstlisting}



\subsection{Comunicaciones}

En el punto anterior hemos visto las diferentes configuraciones de Entrada/Salida establecidas para el control de los diferentes motores y sensores. En el presente y sucesivos puntos describiremos los diferentes
canales de comunicación abiertos y los flujos de información existentes. En la figura \ref{figura:comunicaciones-robot} se muestra un gráfico representativo de los diferentes flujos de datos existentes:


\begin{figure}[H]
  \begin{center}
    \includegraphics[scale=0.6]{diagramas/flujo-comunicaciones-robot.png}
  \end{center}
  \caption{Canales de comunicación abiertos por el robot.}
  \label{figura:comunicaciones-robot}
\end{figure}


Primeramente al accionar el robot, éste envía una notificación al servidor indicando que se encuentra en modo \emph{online} y permanece a la espera de que algún usuario de la aplicación decida conectarse para su control.
Canal de comunicación representado en amarillo en la figura \ref{figura:comunicaciones-robot}.\\

Un usuario ve disponible el robot y decide conectarse. Entonces se establece un enlace entre el robot, servidor, y el usuario, cliente. Flujo de comunicación representado en negro, figura \ref{figura:comunicaciones-robot}.\\

El usuario envía comandos al robot a través del canal abierto y las respuestas son enviadas a al cliente, transmisiones representados en verde para datos y en azul para el vídeo, figura \ref{figura:comunicaciones-robot}.\\

Las tres vías de comunicación entre el robot y el usuario discurren dentro de un mismo canal de comunicación (socket). En el siguiente punto se describirá más detalladamente el procedimiento.

\subsubsection{ Formato del mensaje}



MOTOR-
SRV-


\subsubsection{Sockets}

Ahora bien, una vez definidos los pines que activarán nuestros motores y los canales de comunicación necesitamos que éstos sean activados cuando desde una red externa lo indiquemos haremos uso
de la biblioteca Socket.io. Comenzamos el código incluyendo las librerías necesarias:

\begin{lstlisting}[language=JavaScript]
  var io_client = require('./node_modules/socket.io-client');
  var sails_client = require('./node_modules/sails.io.js');
\end{lstlisting}

Para la comunicaciones usuario y dispositivo robótico se ha empleado la biblioteca socket.io-client.\\

Para las comunicaciones directas con el servidor se ha empleado el SDK\footnote{Un kit de desarrollo de software o SDK (siglas en inglés de software development kit) es generalmente un conjunto de herramientas
de desarrollo de software que le permite al programador o desarrollador de software crear aplicaciones para un sistema concreto, por ejemplo ciertos paquetes de software, frameworks, etc.} proporcionado por el framework
para comunicarse con Sails a través de sockets desde una aplicación Node.js o desde el propio navegador.\\


El siguiente paso una vez cargadas las librerías es establecer las diferentes conexiones. El primer comportamiento deseado es, por parte del robot, lanzar un mensaje al servidor indicando su disponibilidad al servidor con la finalidad de
enviar las diferentes notificaciones a los usuarios que estén usando la aplicación. Para ello establecemos la conexión y enviamos un mensaje al servidor con el identificador del robot junto con su estado online igual a true. A continuación mostramos
el código:\\

\begin{lstlisting}[language=JavaScript]
  var io_server = sails_client(io_client);
  io_server.sails.url = 'http://46.101.102.33:80';
  io_server.socket.get('/robot/changetoonline/', {robot: '59188631c8e94ba54f7a4bdc', online: true});
\end{lstlisting}



Una vez realizado el paso anterior, tenemos un robot que ha indicado a la aplicación web principal de que se encuentra en estado online pero nada más. El siguiente paso sería establecer alguna comunicación que se mantuviera a la escucha a 
la espera de la llegada de nuevas conexiones. Para la creación del socket basta con la siguiente instrucción, la cual recibe un puerto que utilizará para mantenerse a la escucha:\\

\begin{lstlisting}[language=JavaScript]
  var io = require('./node_modules/socket.io').listen(8085, { log: false });
\end{lstlisting}


Con la finalidad de ir capturando los diferentes eventos, se han definido las siguientes funciones para la conexión y desconexión de los clientes:\\

\begin{lstlisting}[language=JavaScript]

io.sockets.on('connection', function (socket)
{

  //Almacenamiento del número total de clientes conectados.
  sockets[socket.id] = socket;
  console.log("Total clientes conectados : ", Object.keys(sockets).length);
  
  //Envío de un saludo.
  socket.emit('robotmsg', {msg: "!!!HOLA!!!"});


  //Salida de un cliente.
  socket.on('disconnect', function() {
    console.log('Bye!');
    stopStreaming(socket);
  });  
  
}
\end{lstlisting}

Cuando un evento \emph{action} es recibido se activada la función que procesa el comando recibido y activa las salidas correspondientes, la cual establece los pines necesarios a los valores
1 o 0 según el parámetro establecido. La tabla \ref{table:table-pin-out} muestra las diferentes combinaciones de salidas y su acción correspondiente:

\begin{table}[H]
  \begin{center}
    \begin{tabular}{|p{2.5cm}|p{2.5cm}|p{2.5cm}|p{2.5cm}|p{2.5cm}|}
      \hline
      {\textbf{Acción}} & \textbf{ GPIO 2 } & \textbf{ GPIO 3 } & \textbf{  GPIO 17 } & \textbf{ GPIO 27 }\\
      \hline
      { \textbf{ UP } } & { 1 } & { 0 }  & { 1 }  & { 0 }  \\
      \hline
      { \textbf{ DOWN } } & { 0 } & { 1 }  & { 0 }  & { 1 } \\
      \hline
      { \textbf{ LEFT } } & { 1 } & { 0 }  & { 0 }  & { 0 } \\
      \hline
      { \textbf{ RIGHT } } & { 0 } & { 0 }  & { 1 }  & { 0 } \\
      \hline
      { \textbf{ STOP } } & { 0 } & { 0 }  & { 0 }  & { 0 }  \\
     \hline   
    \end{tabular}
  \end{center}
\caption{ Combinaciones de salida para los puertos GPIO y su acción correspondiente. }
\label{table:table-pin-out}
\end{table}

A continuación se muestra el ejemplo desarrollado para la activación de los motores en sentido de giro y dirección según la tabla anterior:\\

\begin{lstlisting}[language=JavaScript]
  // Escucha de comandos.  
  socket.on('action', function (data){

    console.log('Comando recibido: ' + data);

    switch(data) {
      case 'UP':
        gpio2.digitalWrite(1);
        gpio3.digitalWrite(0);
        gpio17.digitalWrite(1);
        gpio27.digitalWrite(0);
        console.log('UP');
        break;

      case 'RIGHT':
        gpio2.digitalWrite(0);
        gpio3.digitalWrite(0);
        gpio17.digitalWrite(1);
        gpio27.digitalWrite(0);
        console.log('UP');
        break;

      case 'LEFT':
        gpio2.digitalWrite(1);
        gpio3.digitalWrite(0);
        gpio17.digitalWrite(0);
        gpio27.digitalWrite(0);
        console.log('UP');
        break;

      case 'DOWN':
        gpio2.digitalWrite(0);
        gpio3.digitalWrite(1);
        gpio17.digitalWrite(0);
        gpio27.digitalWrite(1);
        console.log('UP');
        break;

      case 'STOP':
        gpio2.digitalWrite(0);
        gpio3.digitalWrite(0);
        gpio17.digitalWrite(0);
        gpio27.digitalWrite(0);
        console.log('UP');
        break;

      default:
        console.log('command not found');
    }

  })
    
\end{lstlisting}


Diagrama de bloques del robot controlado vía WiFi:

\begin{figure}[H]
  \begin{center}
    \includegraphics[scale=0.3]{diagramas/diagrama_bloques_robot.png}
  \end{center}
  \caption{Diagrama de bloques del robot controlado por WiFi.}
  \label{figura:diagrama-bloques-robot}
\end{figure}


Componentes necesarios para el desarrollo del Robot controlado vía WiFi

\begin{itemize}
 \item Microcontrolador
 \item Raspberry Pi con WiFi incorporado
 \item Sensor de temperatura
 \item Sensor de humerdad.
 \item Sensor de sonidos de baja intensidad.
 \item Sensor de sonidos de elevada intensidad.
 \item Sensor de humos.
 \item Sensor de llamas.
 \item Otros sensores...
 \item Controladora de motores L289N.
 \item Cámara USB.
 \item Buzzer
\end{itemize}


\subsubsection{ Streaming de vídeo }

Para realizar la transferencia de vídeo desde el robot hacia el cliente se ha empleado la librería FFmpeg \footnote{ Los conocimientos necesarios para la comprensión de la herramienta FFmepg y sus modos de 
utilización se han adquirido accediendo a la documentación disponible en la referencia \cite{website:7} correspondiente con la documentación oficial.} haciendo uso de su herramienta de línea de comandos.\\

El procedimiento de captura de vídeo y su posterior transmisión es realizado mediante la siguiente instrucción de Ffmpeg:\\

\begin{lstlisting}[language=bash]
  ffmpeg -f video4linux2 -i /dev/video0 -s 300x150 -f mjpeg pipe:1 -b:v 28k -bufsize 28k
\end{lstlisting}

En los puntos sucesivos analizaremos qué es lo que realiza la instrucción anterior y por qué resulta clave en todo el proceso de difusión, comprendido desde la captura del 
vídeo hasta su posterior transmisión al usuario que está controlando el robot.\\

Nada prodríamos transmitir si no disponemos inicialmente de los datos que queremos difundir. De ahí que inicialmente debamos realizar la captura de las diferentes imágenes a partir de la cámara
USB conectada a la Raspberry Pi. Para ello se utiliza la API de captura de vídeo video4linux2 \footnote{ Video4Linux o V4L es una API de captura de video para Linux. Muchas webcams USB, sintonizadoras
de tv, y otros periféricos son soportados. Video4Linux está integrado con el núcleo Linux. V4L está en su segunda versión (V4L2). El V4L original fue incluido en el ciclo 2.1.X de desarrollo del
núcleo Linux. Video4Linux2 arregla algunos fallos y apareció en los núcleos 2.5.X. } (o simplemente v4l2) la cual dispone las bibliotecas de Ffmpeg. Tan solo debemos especificar el dispositivo de 
captura.\\

El nombre del dispositivo de captura es un nodo de dispositivo de archivo, por lo general los sistemas Linux tienden a crear automáticamente estos nodos cuando el dispositivo
está conectado al sistema, y ​​tiene un nombre del tipo /dev/videoN, donde N es un número asociado al dispositivo.\\ 


Para proceder a la captura de las imágenes nos bastaría con introducir el siguiente comando:\\

\begin{lstlisting}[language=bash]
  ffmpeg -f video4linux2 -i /dev/video0 -s 300x150 -f mjpeg video_out.mpeg
\end{lstlisting}

Ahora bien, el comando anterior toma las imágenes de la cámara y las almacena en el archivo especificado \emph{ video\_out.mpeg} especificando una resolución de salida de 300x150 píxeles
empleando la opción -s. Pero nosotros no deseamos exactamente ese comportamiento. Debemos canalizar esos datos capturados hacia el socket creado con la finalidad de ir transmitiendo los 
diferentes frames y no almacenándolos en disco tal y como realiza la instrucción anterior.\\


Para resolver este problema podemos emplear el sistema de tuberías que implementan los sistemas UNIX \footnote{ Una tubería (pipe, cauce o '|') consiste en una cadena de 
procesos conectados de forma tal que la salida de cada elemento de la cadena es la entrada del próximo. Permiten la comunicación y sincronización entre procesos. Es común el uso de
buffer de datos entre elementos consecutivos. }.

En cualquier sistema Unix se puede hacer que la salida de una determinada orden sea la entrada estándar de otra, lo que le confiere a las órdenes Unix una enorme potencia.
Para realizar dicha "canalización`` debemos utilizar las siguientes opciones:\\

\begin{lstlisting}[language=bash]
  pipe:1 -b:v 28k -bufsize 28k
\end{lstlisting}


Con la opción \emph{pipe:1} accedemos al protocolo pipe de UNIX, el cual lee y escribe de las \emph{tuberias} UNIX siendo el número 1 la tubería correspondiente a la salida estándar 
stdout ( 0 para stdin y 2 para stderr), la cual podría ser omitida puesto que es la salida por defecto.\\

La opción \emph{-b:v 28k} establece la tasa de transferencia, en nuestro caso una tasa de 28 kbit/s.\\

La opción \emph{-bufsize 28k} establece un tamaño de buffer \footnote{Un buffer de datos es un espacio de la memoria en un disco o en un instrumento digital reservado para el almacenamiento
temporal de información digital, mientras que está esperando ser procesada.} de 28 kbits.\\

A continuación mostramos el código de transmisión de vídeo al completo junto con la captura de los diferentes eventos activados cuando se produce la salida de datos por cada una de las salidas estándar:\\

\begin{lstlisting}[language=JavaScript]

  function startStreaming(socket) {
    //ffmpeg -f video4linux2 -i /dev/video0 -s 300x150 -f mjpeg pipe:1 -b:v 28k -bufsize 28k

    if (running_camera == false){
      console.log('Starting streaming....');
      var args = ["-f", "video4linux2", "-i", "/dev/video0", "-s", "300x150","-f","mjpeg", "pipe:1", "-b:v 28k", "-bufsize 28k"]
      ffmpeg_command = require('child_process').spawn("ffmpeg", args);
      running_camera = true
    }

    ffmpeg_command.on('error', function(err, stdout, stderr) {
      console.log("ffmpeg stdout:\n" + stdout);
      console.log("ffmpeg stderr:\n" + stderr);
      running_camera = false
    });


    ffmpeg_command.on('close', function (code) {
      console.log('ffmpeg exited' + code );
      running_camera = false
    });


    ffmpeg_command.stderr.on('data', function (data) {
      //console.log('stderr: ' + data);
    });

    ffmpeg_command.on('end', function() {
      console.log('Finished');
      running_camera = false
    });

    ffmpeg_command.stdout.on('data', function (data) {
      //console.log('stdout: ' + data);
      var frame = new Buffer(data).toString('base64');
      socket.emit('canvas',frame);
    });
  }

\end{lstlisting}


\subsubsection{Código de ejemplo completo}

Finalmente se muestra el código completo para el robot de pruebas desarrollado. Dicho código puede emplearse como guía de referencia o plantilla para futuros proyectos con la idea de integrarlos en la aplicación RobotUI.\\


\begin{lstlisting}[language=JavaScript]
var io_client = require('./node_modules/socket.io-client');
var sails_client = require('./node_modules/sails.io.js');
var io_server = sails_client(io_client);
io_server.sails.url = 'http://46.101.102.33:80';
io_server.socket.get('/robot/changetoonline/', {robot: '59188631c8e94ba54f7a4bdc', online: true});

// Inicia servidor socket.io en el puerto 8085.
var io =io_client.listen(8085, { log: false });

// Carga de módulos necesarios.
var ffmpeg_command, running_camera = false, child_process = require('child_process');

var Gpio = require('pigpio').Gpio;
// Pines utilizados. Motores izquierdos: 2 y 3, motores derechos: 17 y 27
var gpio2 = new Gpio(2, {mode: Gpio.OUTPUT}),
  gpio3 = new Gpio(3, {mode: Gpio.OUTPUT}),
  gpio17 = new Gpio(17, {mode: Gpio.OUTPUT}),
  gpio27 = new Gpio(27, {mode: Gpio.OUTPUT});


console.log('Esperando conexión...');

var sockets = {};

io.sockets.on('connection', function (socket)
{

  sockets[socket.id] = socket;
  console.log("Clientes totales conectados: ", Object.keys(sockets).length);

  socket.on('disconnect', function() {
    console.log('¡Adios!');
    //stopStreaming(socket);
  });


  socket.on('start-stream', function() {
    startStreaming(socket);
  });

  socket.emit('robotmsg', {msg: "¡¡¡Bienvenido!!!"});
  console.log('emitiendo: ' + "¡¡¡Bienvenido!!!");

  socket.on('action', function (data){

    console.log('Comando recibido: ' + data);

    switch(data) {
      case 'UP':
        gpio2.digitalWrite(1);
        gpio3.digitalWrite(0);
        gpio17.digitalWrite(1);
        gpio27.digitalWrite(0);
        console.log('UP');
        break;

      case 'RIGHT':
        gpio2.digitalWrite(0);
        gpio3.digitalWrite(0);
        gpio17.digitalWrite(1);
        gpio27.digitalWrite(0);
        console.log('UP');
        break;

      case 'LEFT':
        gpio2.digitalWrite(1);
        gpio3.digitalWrite(0);
        gpio17.digitalWrite(0);
        gpio27.digitalWrite(0);
        console.log('UP');
        break;

      case 'DOWN':
        gpio2.digitalWrite(0);
        gpio3.digitalWrite(1);
        gpio17.digitalWrite(0);
        gpio27.digitalWrite(1);
        console.log('UP');
        break;

      case 'STOP':
        gpio2.digitalWrite(0);
        gpio3.digitalWrite(0);
        gpio17.digitalWrite(0);
        gpio27.digitalWrite(0);
        console.log('UP');
        break;

      default:
        console.log('command not found');
    }

  })
});

function stopStreaming(socket) {
  delete sockets[socket.id];
  // no more sockets, kill the stream
  if (Object.keys(sockets).length == 0) {
    if (ffmpeg_command){
      ffmpeg_command.kill();
      running_camera = false;
      console.log('Stop streaming');
    }
  }
}

function startStreaming(socket) {
  //ffmpeg -f video4linux2 -i /dev/video0 -s 300x150 -f mjpeg pipe:1 -b:v 28k -bufsize 28k

  if (running_camera == false){
    console.log('Starting streaming....');
    var args = ["-f", "video4linux2", "-i", "/dev/video0", "-s", "300x150","-f","mjpeg", "pipe:1", "-b:v 28k", "-bufsize 28k"]
    ffmpeg_command = child_process.spawn("ffmpeg", args);
    running_camera = true
  }

  ffmpeg_command.on('error', function(err, stdout, stderr) {
    console.log("ffmpeg stdout:\n" + stdout);
    console.log("ffmpeg stderr:\n" + stderr);
    running_camera = false
  });


  ffmpeg_command.on('close', function (code) {
    console.log('ffmpeg exited' + code );
    running_camera = false
  });


  ffmpeg_command.stderr.on('data', function (data) {
    //console.log('stderr: ' + data);
  });

  ffmpeg_command.on('end', function() {
    console.log('Fin');
    running_camera = false
  });

  ffmpeg_command.stdout.on('data', function (data) {
    //console.log('stdout: ' + data);
    var frame = new Buffer(data).toString('base64');
    socket.emit('canvas',frame);
  });

}

\end{lstlisting}

Para la ejecución del código introducimos el siguiente comando:

\begin{lstlisting}[language=bash]
  sudo node raspberry.js
\end{lstlisting}

Siendo \emph{raspberry.js} el nombre del archivo que contiene nuestro código.



