% Este archivo es parte de la memoria del proyecto fin de carrera
% de Manuel López Urbina. Protegida bajo la licencia GFDL.
% Para más información, la licencia completa viene incluida en el
% fichero fdl-1.3.tex

% Copyright (C) 2018 Manuel López Urbina

\newpage

\chapter{Comentarios finales}
\label{chap:conclusiones}

\section{Presupuesto}

\begin{table}[H]
  \begin{center}
    \begin{tabular}{|p{8cm}|p{2cm}|p{2cm}|p{2cm}|}
      \hline
      \vspace{+0.2in}{\textbf{Descripción}} & {\textbf{Unidades}} & {\textbf{Precio \EUR{} unidad }} & {\textbf{Total \EUR{}}}\\
      \hline
      \vspace{+0.2in}{Raspberry Pi 3 Modelo B} &  \vspace{+0.2in}{1} &  \vspace{+0.2in}{38,70} &  \vspace{+0.2in}{38,70}\\
      \hline
      \vspace{+0.2in}{Arduino Mega} & \vspace{+0.2in}{1} &  \vspace{+0.2in}{38,70} &  \vspace{+0.2in}{35,98}\\
      \hline      
      \vspace{+0.2in}{Arduino Sensor kit} & \vspace{+0.2in}{1} &  \vspace{+0.2in}{38,70} &  \vspace{+0.2in}{25,00}\\
      \hline
      \vspace{+0.2in}{Cámara USB alta definición} &  \vspace{+0.2in}{1} &  \vspace{+0.2in}{39,90} &  \vspace{+0.2in}{39,90}\\
      \hline
      \vspace{+0.2in}{Tarjeta de expansión con batería de Litio para Raspberry Pi} &  \vspace{+0.2in}{1} &  \vspace{+0.2in}{16,99} &  \vspace{+0.2in}{16,99}\\
      \hline
      \vspace{+0.2in}{Lipo batería (3.7v, 600mAh Lipo) } & \vspace{+0.2in}{1} & \vspace{+0.2in}{13,99} & \vspace{+0.2in}{13,99}\\
      \hline
      \vspace{+0.2in}{Indicador Tester de baterías Lipo} & \vspace{+0.2in}{1} & \vspace{+0.2in}{8,90} & \vspace{+0.2in}{8,90}\\
      \hline      
      \vspace{+0.2in}{Cargador baterías Lipo} & \vspace{+0.2in}{1} & \vspace{+0.2in}{21,90} & \vspace{+0.2in}{21,90}\\
      \hline
      \vspace{+0.2in}{Chasis vehículo Radiocontrol} & \vspace{+0.2in}{1} & \vspace{+0.2in}{54} & \vspace{+0.2in}{54}\\
      \hline
      \vspace{+0.2in}{Instancia Amazon Web Service} & \vspace{+0.2in}{2 meses} & \vspace{+0.2in}{5/mes} & \vspace{+0.2in}{10}\\
      \hline        
      \vspace{+0.2in}{Horas de programación y montaje} & \vspace{+0.2in}{350 horas} & \vspace{+0.2in}{50/hora} & \vspace{+0.2in}{17500}\\
      \hline
    \end{tabular}
  \end{center}
\end{table}

\begin{table}[H]
  \begin{flushright}
    \begin{tabular}{p{8cm}p{2cm}}
      \vspace{+0.1in}\textbf{Total bruto:} &\vspace{+0.1in}{\EUR{17745,38}}\\
      \vspace{+0.1in}\textbf{I.V.A. \%: } & \vspace{+0.1in}{21\%}\\
      \vspace{+0.2in}\textbf{Total presupuesto:} & \vspace{+0.2in}{\EUR{21471,91}}\\
    \end{tabular}
  \end{flushright}
\end{table}


\section{Conclusiones}

La elaboración de este proyecto ha resultado muy gratificante a nivel personal. Uno de los motivos principales ha sido la necesidad de trabajar en numerosas áreas de 
conocimiento entre las que encontramos, por un lado la programación, ya que ha sido necesaria la programación del microcontrolador que incorpora la placa Arduino, 
concretamente el ATmega2560. Por otra parte se ha realizado la programación de la placa Raspberry Pi, módulo encargado de la comunicación por puerto serie con la placa Arduino y la 
transmisión de los datos al servidor. 

Por otro lado, la elaboración del vehículo robótico me ha permitido ampliar conocimientos en áreas más relacionadas con la electrónica, un área bastante desconcida para mí. 
Donde se ha trabajado, entre otras áreas, en lo referente a la transmisión de señales por puerto serie, la emisión de pulsos PWM junto con la utilización de multitud de sensores.\\

Entre los elementos desarrollados podemos destacar:\\

\begin{itemize}
 \item La elaboración de un vehículo roboótico haciendo uso de una Raspberry Pi 3 Model B.
 \item Programación del micronoctrolador Atmega2560 alojado en una placa Arduino.
 \item Comunicación por puerto serie entre la placa Arduino y Raspberry Pi
 \item Realización de conexiones entre elementos y montaje del vehículo.
 \item Transmisión de gran cantidad de datos entre cliente servidor y servidor cliente. Streaming de vídeo y audio, emisión de comandos entre otros datos.\\
 \item Incorporación de algún sistema para el geoposicionamiento y escaneado de habiataciones, obstáculos, etc.
\end{itemize}

Pienso que el presente royecto puede ser de gran utilidad permitiendo liberar a los trabajadores de aquellas tareas especialmente peligrosas para la salud y la integridad física en diversas situaciones 
como exploración de zonas peligrosas o de difícil acceso, búsqueda de personas, etc ya que el vehículo desarrolado permite introducirlo de manera remota en zonas inaccesibles 
o peligrosa para las personas.\\

Una vez presentado podré continuar añadiendo mejoras y muchas otras cosas que tengo pensadas y que, posiblemente, se realicen a modo proyecto personal y ocio.\\

\section{Mejoras futuras}

La aplicación puede mejorarse en diversos aspectos. A continuación, se citan algunas de las mejoras que pueden llevarse a cabo:

\begin{itemize}
  \item Incrementar el número de sensores en el sistema que permitan captar nuevos parámetros o situaciones.
  \item Dotar del sistema de la posibilidad de otras posibilidades de cnectividad distintas al WiFi como 4G, buetooth, etc.
  \item Incorporar una cámara con posibilidad de visión 360 grados.
  \item Añadir autenticación por un sistema de certificados por clave pública y privada que garantice que la conectividad entre dispositivo robótico y usuario es la correcta.
\end{itemize}

