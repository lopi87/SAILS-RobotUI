% Este archivo es parte de la memoria del proyecto fin de carrera
% de Manuel López Urbina. Protegida bajo la licencia GFDL.
% Para más información, la licencia completa viene incluida en el
% fichero fdl-1.3.tex

% Copyright (C) 2018 Manuel López Urbina

\newpage

\chapter{ Pruebas }
\label{chap:pruebas}

La realización de pruebas de software, es una de las fases del ciclo del software de gran importancia, la cual permite la identificación de posibles fallos en la implementación, 
calidad o usabilidad de la aplicación con la finalidad de que el producto final cumpla una serie de garantías y sea de calidad.\\

\section{Plan de pruebas}

Debido a la arquitectura de SensorRS y a la metodología basada en desarrollo incremental que se ha seguido en el desarrollo y construcción del proyecto robótico,
se ha establecido un plan de pruebas en el que las diferentes partes han ido siendo analizadas y testeadas de forma independiente. La aplicación y el robot ha sido sometido 
a las siguientes pruebas:\\

\begin{description} 

\item [Pruebas de funcionamiento:]\\

\begin{itemize}
    \item[]
 \item Se controla que los diferentes elementos y componentes electrónicos se encuentran correctamente alimentados y con un conexionado lo suficientemente fuerte y estable 
 para que no de lugar a posibles futuras fallas.
\end{itemize}


\item [Pruebas de comunicaciones:]\\

\begin{itemize}
    \item[]
 \item Se comprueba que se realiza una comunicación correcta entre la placa Arduino - Raspberry Pi y entre Raspberry PI - Servidor.
\end{itemize}


\item [Pruebas de control:] \\

\begin{itemize}
    \item[]
 \item Se comprueba que se realiza un manejo adecuado del dispositivo robótico.
 \item Se comprueba que se obtiene y muestra correctamente las diferentes mediciones obtenidas de los sensores.
\end{itemize}

\end{description}