% Este archivo es parte de la memoria del trabajo fin de máster
% de Manuel López Urbina. Protegida bajo la licencia GFDL.
% Para más información, la licencia completa viene incluida en el
% fichero fdl-1.3.tex

% Copyright (C) 2018 Manuel López Urbina

\section*{Resumen}
\label{resumen}

Multi Sensor Robot System (SensorRS), Vehículo robótico multisensorial de exploración controlado por wifi basado en Arduino y Raspberry Pi, es un proyecto robótico elaborado
principalmente con una placa Raspberry Pi interconectada por puerto serie con una placa Arduino. \\

La placa Arduino se ha utilizado para la conexión de una serie de sensores cuya
finalidad es la obtención de variables del entorno como temperatura, humedad, iluminación, etc. Estos parámetros son obtenidos mediante la programación del microcontrolador 
que la placa Arduino incorpora, encargándose, posteriormente, de transmitir información a la placa Raspberry Pi para su procesamiento y envío al servidor web donde se realiza
el seguimiento y control del vehículo y los parámetros obtenidos.\\

La intención principal de este trabajo es la de permitir que usuarios puedan disponer de un vehículo robótico a coste muy reducido y de componentes de fácil adquisición y
de reducido coste. Este vehículo podrá ser utilizado para la realización de labores de exploración de zonas en las que o bien no puede acceder una persona debido al reducido tamaño o 
dificultoso acceso de las áreas a explorar o bien porque alguna situación peligrosa lo impida. En resumidas cuentas un vehículo robótico con un amplio sistema de telemetría incorporado.\\

Dicho sistema es configurado en la aplicación web RobotUI para permitir sus control por parte de otros usuarios. RobotUI incorpora un asistente mediante el cual, cualquier usuario 
sin conocimientos previos de programación, pueda elaborar una interfaz para el control de su dispositivo robótico personalizada y adaptada a sus necesidades de control 
y a las características del dispositivo en cuestión.\\

Dicha interfaz permitirá realizar el control de los mencionados dispositivos además de permitir que otros usuarios entren en las salas o canales donde podrán visualizar el manejo 
que realiza un usuario de su robot en tiempo real donde obtendrán en todo momento las imágenes captadas junto con los diferentes comandos accionados por el usuario 
que está realizando el control.\\

Con todo ello, lo que se busca es crear un sistema robótico controlable por el usuario y que transmita información en tiempo real destinado a la exploración de áreas peligrosas y 
de coste reducido.\\

\textbf{Palabras clave:} Internet, aplicación web, robótica, robots, interfaz de usuario, streaming de vídeo, control remoto, tiempo real, Raspberry Pi, Arduino, sensores, comunicación serie,
telemetría.\\
