% Este archivo es parte de la memoria del proyecto fin de carrera
% de Manuel López Urbina. Protegida bajo la licencia GFDL.
% Para más información, la licencia completa viene incluida en el
% fichero fdl-1.3.tex

% Copyright (C) 2018 Manuel López Urbina

\section*{Resumen}
\label{resumen}

SensorRS, Multi-sensor Robot System, es un proyecto robótico elaborado principalmente con una placa Raspberry Pi interconectada por pueto serie con una placa Arduino. La placa Arduino
se ha utilizado para la conexión de una serie de sensores cuya finalidad es la obtención de variables del entorno como temperatura, humedad, iluminación, etc. Estos parámetros 
son obtenidos mediante la programación del microcontrolador que dicha placa incorpora encargándose, posteriormente, de transmitir dicha información a la placa Raspberry Pi
para su procesamiento y envío al servidor web para su seguimiento y control del vehículo.\\

La intención principal de este proyecto es la de permitir que usuarios dispongan de un vehículo robótico para el acceso y exploración de 
zonas en las que o bien no puede acceder una persona debido al reducido tamaño o dificultoso acceso de las áreas a explorar o 
bien porque alguna situación peligrosa lo impida. En resumidas cuentas un sistema de telemetría.\\

Dicho sistema es configurado en la aplicación web RobotUI para permitir sus control por parte de otros usuarios. RobotUI incorpora un asistente mediante el cual, cualquier usuario 
sin conocimientos previos de programación, pueda elaborar una interfaz para el control de su dispositivo robótico, personalizada y adaptada a sus necesidades de control 
y a las características del dispositivo en cuestión.\\

Dicha interfaz permitirá realizar el control de los mencionados dispositivos además de permitir que otros usuarios entren en las salas o canales donde podrán visualizar el manejo 
que realiza un usuario de su dispositivo robótico en tiempo real donde obtendrán en todo momento las imágenes captadas junto con los diferentes comandos accionados por el usuario 
que está realizando el control.\\

Con ello, lo que se busca es crear un sistema robótico controlable por el usuario y que transmita información en tiempo real destinado a la exploración de áreas peligrosas, 
de fácil consturcción y de coste reducido.\\

\textbf{Palabras clave:} Internet, aplicación web, robótica, robots, interfaz de usuario, streaming de vídeo, control remoto, tiempo real, Raspberry Pi, Arduino, sensores, comunicación serie,
telemetría.\\
